\documentclass[main.tex]{subfiles}

\begin{document}
\section{Introduction}
In high energy particle collisions, hadrons with heavy quark content, are a uniquely versatile probe of of fragmentation dynamics. Their defining feature, a charm ($\mrm{c}$) or bottom ($\mrm{b}$) 
flavoured quark, cannot originate from the hadronization process, but must be created either in the hard process, or in the parton shower, both calculable with perturbative techniques. 
As opposed to the even heavier top ($\mrm{t}$) quark, hadrons with $\mrm{c}$ -and $\mrm{b}$--type content are, however, still understood to fragment through the same mechanisms as their 
light counterparts, $\mrm{u}$, $\mrm{d}$ and $\mrm{s}$. Usually, two quite different (and thus complementary) 
techniques are used to predict yields of charm hadrons. One is the factorisation approach \cite{}, where the cross section is separated into a convolution of three factors: 1) a Parton Distribution 
Function (PDF) of the incoming hadron, 2) the parton level hard scattering cross section, where state--of--the--art calculations today are next--to--leading--order in the strong coupling 
($\alpha_s$) with next--to--leading--log resummation techniques applied, such as \eg GM--VFNS \cite{} or FONLL \cite{}, and finally 3) fragmentation functions, analytical expressions fitted to 
\epem and \ep data \cite{} giving differential probabilities for the charm quark to fragment to various hadron species. (Insert here statments about how it is silly to assume independent fragmentation
when we know it does not work, and how it is known that there are no universal fragmentation functions for charm since long)

Another approach is given by Monte Carlo event generators, such as \pythia \cite{}. PDFs are still used to extract participating partons from the colliding 
%\begin{figure}[t!]
%\begin{minipage}[c]{0.49\linewidth}
%\centering
%\includegraphics[width=\linewidth]{figures/}}\\
%(a)
%\end{minipage}
%\begin{minipage}[c]{0.49\linewidth}
%\centering
%\includegraphics[width=\linewidth]{figures}\\
%(b)
%\end{minipage}\\
%\caption{Caption goes here.}
%\label{fig:}
%\end{figure}

\end{document}


