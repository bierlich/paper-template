\documentclass[main.tex]{subfiles}

\begin{document}
\section{Synopsis}
(This synopsis is to be removed before submission)
In this paper the aim is to explore to what degree junctions is more than a one--trick pony
for charm baryons. The aim is to a) give a more reasonable description of at least XiC (and possibly also SigmaC and OmegaC), while maintaining the good description of Lambda. b) Furthermore similar plots in pPb (not PbPb neccesarily) should be shown. 

The main story of the paper is suggested to be, that while fragmentation approaches have served us well for long enough, these charm baryon data from LHC should really tell us that dynamical models is the way forward -- especially if we want a unified picture for all collision systems.

\section{Introduction}
\label{sec:introduction}
In high energy particle collisions, hadrons with heavy quark content, are a uniquely versatile probe of of fragmentation dynamics. Their defining feature, a charm ($\mrm{c}$) or bottom ($\mrm{b}$) 
flavoured quark, cannot originate from the hadronization process, but must be created either in the hard process, or in the parton shower, both calculable with perturbative techniques. 
As opposed to the even heavier top ($\mrm{t}$) quark, hadrons with $\mrm{c}$ -and $\mrm{b}$--type content are, however, still understood to fragment through the same mechanisms as their 
light counterparts, $\mrm{u}$, $\mrm{d}$ and $\mrm{s}$. Usually, two quite different (and thus complementary) 
techniques are used to predict yields of charm hadrons. One is the factorisation approach \cite{Collins:1985gm,Collins:1989gx}, where the cross section is separated into a convolution of three factors: 1) a Parton Distribution 
Function (PDF) of the incoming hadron, 2) the parton level hard scattering cross section, where state--of--the--art calculations today are next--to--leading--order in the strong coupling 
($\alpha_s$) (see \eg \cite{Nason:1987xz,Nason:1989zy,Mangano:1991jk}) often with next--to--leading--log resummation techniques applied as well, such as \eg GM--VFNS \cite{Helenius:2018uul} or FONLL \cite{Cacciari:1998it,Cacciari:2012ny}, and finally 3) fragmentation functions, analytical expressions fitted to 
\epem and \ep data \cite{} giving differential probabilities for the charm quark to fragment to various hadron species. 

Another approach is given by Monte Carlo event generators, such as \pythia \cite{Bierlich:2022pfr}. PDFs are still used to extract participating partons from the colliding nucleons, but where the focus of the factorisation approach tends to be more directed towards more formal precision in the calculation of the hard scattering, the focus of the Monte Carlo generators are more towards coherent modeling of both perturbative and non-perturbative aspects, such as hadronization.

For charm production in particular, 


\section{Charm hadron production in \pythia}

\subsection{Hard process and parton shower production}
Here we write a brief summary of Pythia hard production of charm from \cite{Norrbin:1998bw}.

\subsection{Heavy ion collisions with \angantyr}
Brief summary, mainly to state that we are dealing with separate subcollisions stacked on top of each other. Harsh could possibly write this.

\subsection{Hadronization and colour reconnection}
Summary of how things have been done so far.

\section{Improvements to the junction model}
Description of Harsh's improvements to junction hadronization. If you already have a paper out, it could mostly be referencing that. Harsh should write this.

\section{Results}

\subsection{Results for proton--proton collisions}

\subsection{Results for proton--lead collisions}

\section{Conclusion}

%\begin{figure}[t!]
%\begin{minipage}[c]{0.49\linewidth}
%\centering
%\includegraphics[width=\linewidth]{figures/}}\\
%(a)
%\end{minipage}
%\begin{minipage}[c]{0.49\linewidth}
%\centering
%\includegraphics[width=\linewidth]{figures}\\
%(b)
%\end{minipage}\\
%\caption{Caption goes here.}
%\label{fig:}
%\end{figure}

\end{document}


